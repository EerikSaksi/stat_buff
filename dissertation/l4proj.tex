% REMEMBER: You must not plagiarise anything in your report. Be extremely careful.

\documentclass{l4proj}

    
%
% put any additional packages here
%

\begin{document}

%==============================================================================
%% METADATA
\title{Level 4 Project Report Template}
\author{John H. Williamson}
\date{September 14, 2018}

\maketitle

%==============================================================================
%% ABSTRACT
\begin{abstract}
    Every abstract follows a similar pattern. Motivate; set aims; describe work; explain results.
    \vskip 0.5em
    ``XYZ is bad. This project investigated ABC to determine if it was better. 
    ABC used XXX and YYY to implement ZZZ. This is particularly interesting as XXX and YYY have
    never been used together. It was found that  
    ABC was 20\% better than XYZ, though it caused rabies in half of subjects.''
\end{abstract}

%==============================================================================

% EDUCATION REUSE CONSENT FORM
% If you consent to your project being shown to future students for educational purposes
% then insert your name and the date below to  sign the education use form that appears in the front of the document. 
% You must explicitly give consent if you wish to do so.
% If you sign, your project may be included in the Hall of Fame if it scores particularly highly.
%
% Please note that you are under no obligation to sign 
% this declaration, but doing so would help future students.
%
%\def\consentname {My Name} % your full name
%\def\consentdate {20 March 2018} % the date you agree
%
\educationalconsent


%==============================================================================
\tableofcontents

%==============================================================================
%% Notes on formatting
%==============================================================================
% The first page, abstract and table of contents are numbered using Roman numerals and are not
% included in the page count. 
%
% From now on pages are numbered
% using Arabic numerals. Therefore, immediately after the first call to \chapter we need the call
% \pagenumbering{arabic} and this should be called once only in the document. 
%
% Do not alter the bibliography style.
%
% The first Chapter should then be on page 1. You are allowed 40 pages for a 40 credit project and 30 pages for a 
% 20 credit report. This includes everything numbered in Arabic numerals (excluding front matter) up
% to but excluding the appendices and bibliography.
%
% You must not alter text size (it is currently 10pt) or alter margins or spacing.
%
%
%==================================================================================================================================
%
% IMPORTANT
% The chapter headings here are **suggestions**. You don't have to follow this model if
% it doesn't fit your project. Every project should have an introduction and conclusion,
% however. 
%
%==================================================================================================================================
\chapter{Introduction}

% reset page numbering. Don't remove this!
\pagenumbering{arabic} 



Why should the reader care about what are you doing and what are you actually doing?
\section{Guidance}

\textbf{Motivate} 
Activity trackers are very popular nowadays. These apps leverage various motivators, such as social feeds, points for activity, and reminders. However, most of these apps do not leverage collaborative motivators. I wanted to investigate whether or not users would feel more motivated to exercise if it wasn't just about working towards a personal goal, but towards a shared goal with a team of friends/strangers. The goal of my RPG themed activity tracker, 'Stat Buff' was to gamify progress as ingame character development (leading to greater attack damage), and workouts as attacks towards your teams enemy. This rewards users for making progress in their performance, as well as for making small daily efforts. The teams enemy resets in a week if it has not been defeated, which creates an urgency to exercise in order to to reach the next enemy.
\section{Writing guidance}
\subsection{Who is the reader?}

This is the key question for any writing. Your reader:

Remember, you will be marked by your supervisor and one or more members
of staff. You might also have your project read by a prize-awarding
committee or possibly a future employer. Bear that in mind.

\subsection{References and style guides}

\subsubsection{Citation styles}

%\item If you are referring to a reference as a noun, then cite it as: ``\citet{Orw68} discusses the role of language in political thought.''
%\item If you are referring implicitly to references, use: ``There are many good books on writing \citep{Orw68, Wil09, Pin15}.''

There is a complete guide on good citation practice by Peter Coxhead available here: \url{http://www.cs.bham.ac.uk/~pxc/refs/index.html}. 
If you are unsure about how to cite online sources, please see this guide: \url{https://student.unsw.edu.au/how-do-i-cite-electronic-sources}.

\subsection{Plagiarism warning}

\begin{highlight_title}{WARNING}
    
    If you include material from other sources without full and correct attribution, you are commiting plagiarism. The penalties for plagiarism are severe.
    Quote any included text and cite it correctly. Cite all images, figures, etc. clearly in the caption of the figure.
\end{highlight_title}


%==================================================================================================================================
\chapter{Background}
The most similar research I could find on collaborative activity trackers was a paper on an app named 'Pass the Ball' cite{}. This app was very simple: a team has one ball, and the person with that ball could score points for the team by exercising, and pass it to others. This very simple idea lead to many feeling responsibility to exercise when they had the ball. It also lead to lots of issues: users hogging the ball, stealing the ball for a point penalty, etc. There were also issues for users with irregular schedules. This study made me understand the importance of accountability and structure, but also the possible friction that this can cause if not implemented well. I also looked into a case study on 'Spy Feet' cite{} which was a pedometer that let the user progress in a story through walking. The researchers were I did some research into the effectiveness of activity trackers. Although this was singleplayer and story based, it was reassuring to see that gamified abstractions motivated users to exercise more. I also did some research into the state of activity trackers as a whole. 'Behavior Change Techniques in Top-Ranked Mobile Apps for Physical Activity' cite{} talked about the lack of behavioural change techniques employed in top apps, aswell as the two primary categories: educational and motivational. This made me consider which route I should take (I leaned more towards motivational, with educational aspects). 'Apps of Steel: Are Exercise Apps Providing Consumers With Realistic Expectations?' came to the same conclusion as the earlier paper: apps don't use enough behavioural change techniques. I kept this in mind, but still kept my focus on collaborative aspects. I also read 'Personalizing Mobile Fitness Apps Using Reinforcement Learning', as I was interested in if it was possible to increase the effectiveness of the collaborative techniques through machine learning. This was out of scope to implement due to a lack of experience and time. I read 'A Tale of Two Perspectives: A Conceptual Framework of User Expectations and Experiences of Instructional Fitness Apps' in order to understand what users prioritize in activity trackers. The app found that content was most important for 47.2%, technical implementation and utilities 39.7%, and psychological features were the most important for 13.2% of users. Because of this, I decided that I should pay attention to content and technical details even if they aren't what is being studied, as this will make the app more attractive and usable for participants. Finally, I did some research into behaviour change by looking at 'Integrated Theory of Health Behavior Change'. } I didn't have the time to implement most techniques, but it gave me a good understanding of how social support can be leveraged to drive behavioural change.


I analyzed existing exercise apps, and existing RPG games in order to better understand the two categories I am combining. I analyzed Stronglifts 5x5 and Jefit, two popular exercise apps. Both had inbuilt social features, and online forums to keep people motivated. They had some complicated exercise features like planning workouts, which were out of scope for me. I found that Jefit had competitions, but no collaborative exercising. The absence of collaborative features showed that there was room to explore the effectiveness of these apps. I had to make sure that I implemented at least some of the features of existing popular exercise apps in order to maintain user engagement. These were the key features that I noticed in both these apps.

\begin{itemize}    
    \item
      Social features (adding friends/chatting)
    \item
      Being able to plan workouts
    \item
      Being able to track a wide variety of exercises
    \item 
      Built in exercise tutorials
    \item
      Satisfying animations/congratulating progress
    \item
      Progression suggestions (add more or less weight)
\end{itemize}

From my analysis of RPG games, I realized that collaboration is an integral part of RPG games, and that existing products should serve as a model for implementing collaborative aspects. For instance, Clash of Clans has clans, where users need to give each other troops in order to optimally defend themselves. Clans also battle against one another in "Clan Wars" where users have to attack each others bases in order to win. Wins require members to attack consistently and to reinforce their fellow members \citep{coc}. 
Another example of a collaborative RPG is World of Warcraft. In World of Warcraft, users participate in Raids together, which require very detailed coordination from all members \citep{wow}. There are also multiple different roles which users have to take, such as healer, tank, or DPS (damage dealer) \citep{wow_roles}. Collaboration is so important in this game, that there is a very popular video \citet{leeroy_jenkins} where a team tries to plan for a raid, but a player, named "Leeroy Jenkins" ruins the entire raid by not collaborationg, going in alone forcing the team to join him. The fact that one person was able to ruin a raid, and that millions were able to relate to the frustration of non-collaborating members shows how integral collaboration is in this game. Even many singleplayer RPG games are built around collaboration. For instance the game Darkest Dungeon \citet{darkest_dungeon} is built around controlling your various heroes, who can give each other status bonuses, and heal each other. For instance, the Plague Doctor hero is focused on healing others, while a Highwayman can deal lots of damage. From these three RPG collaborative games, I derived the following key points:

\begin{itemize}    
    \item
      Teams should fight against other players (or non-player characters)
    \item
      Users should be encouraged to help each other
    \item
      Users should have specific roles that serve some purpose in achieving the end goal.
    \item 
      There should be some clear progression: both on an individual level and on a team level. These should be linked (the team should progress faster if individual users are progressing)
    \item 
      There should be accountability: if some user(s) aren't doing their part, they should not progress.
\end{itemize}




\section{Guidance}
\begin{itemize}    
    \item
      Don't give a laundry list of references.
    \item
      Tie everything you say to your problem.
    \item
      Present an argument.
    \item Think critically; weigh up the contribution of the background and put it in context.    
    \item
      \textbf{Don't write a tutorial}; provide background and cite
      references for further information.
\end{itemize}

%==================================================================================================================================
\chapter{Analysis/Requirements}
Exercise is hard to do consistently, leading to high inactivity. Social and group based exercise (such as exercising with a friend or joining a class) are good solution for making people feel accountable, and forcing a schedule. Even then, exercise would still be for your own good, and not for some common goal. A collaborative activity tracker would try to remove the buy in of group based exercise (course cost, fixed schedule) and would make the user feel as though they were contributing to a collective success and not just an individual one.
\section{Guidance}
Make it clear how you derived the constrained form of your problem via a clear and logical process. 

%==================================================================================================================================
\chapter{Design}

\subsection{Calculating a user's contribution}
In order to make the game collaborative, I wanted the users to have some common goal. I figured that a simple non-player character would be the easiest to implement. Team's would progress by defeating an enemy, and progress to the next one. The enemies would progressively have more health. There should be a constant time limit to defeat the enemy before it is reset back to full health, which should hopefully create a sense of urgency, and require users to be consistent with their training to defeat enemies.

I wanted this app to calculate your team's contribution based on your effort and your strength. Rewarding effort should hopefully motivate users to workout regularly and push themselves, even when progress plateaus. Progress is rarely a straight line: it often has bumps, but through perseverance, you progress more than you regress. This also motivates users with lots of experience, as they can often feel demotivated by the waning progress, and the amount of time and effort it can take to accomplish something a beginner can achieve in a fraction of the time. I also reward strength, as I want the app to provide users with long term motivation. I don't want users to simply exercise for the sake of exercise, but also for the long-term improvements in strength. Rewarding progression is also important as it disincentivizes overtraining. If I only reward effort, the optimal strategy would be to exercise as much as possible, even if you feel pain and your progression stops and regresses. By rewarding strength improvements, I reward the user for training moderately and resting adequately, as these are integral for long term progress. I gamify strength as attack damage, and effort as the number of attacks you deliver towards your team's enemy. The product of the two is the total damage that you deal towards your team's enemy. This total damage is dealt towards a monster that your enemy is currently fighting, and represents a members contribution to the group.

Initially I had to quantify strength, or attack damage. I wanted strength to be calculated relative to the person performing it. For example, it is harder for an average 50kg person to lift a certain weight, than for the average 80kg person. By taking this into account, I made the app more inclusive, and allowed everyone to contribute based on how strong they are relative to others like them, and not everyone else. Relative strength is simply a percentile that is calculated based on the demographic of the person, the exercise they performed, and the weight and repetitions that they used. There are sites which perform this calculation, which I will go into more detail in the implementation section. I was still left with the challenge of converting per-exercise percentiles into a single strength value. Initially, I used a simple naive implementation, where I took the mean percentile of all exercises, and used that as the final strength value. This is problematic, as this would encourage users to only track their best exercise, and nothing else. I don't want my app to encourage users to just focus on one exercise, as this can lead to injuries and imbalances. With this in mind, I decided that users should be rewarded for adding more exercises up to a certain number.  I decided to calculate strength by taking the product of the percentile with the number of exercises that you're doing. This seems good in practice, but this is simplifies to be just the sum of percentiles.

\begin{algorithm}
  $\frac{\sum_{i=0}^{n} x_i}{n} * n = {\sum_{i=0}^{n} x_i}$
\end{algorithm}

I only realized this on the first day of the evalution. Luckily, this was a simple database function, so it was easy to replace this calculation. As I wanted users to be rewarded for adding more exercises, but only up to a certain point, I decided to multiply the average strength percentage with the number of exercises log scaled. I think it's important to disincentivize laborious strategies such as keeping track of 50 exercises just because it would increase your total damage. 

\begin{algorithm}
  $attack\_damage = \frac{\sum_{i=1}^{n} x_i}{n} * ln(n) $

  where the user has tracked $n$ exercises, and $x$ contains the strength percentiles for exercises 
\end{algorithm}

Next, I had to decide how I would grant attacks based on workout effort. I could compare the user's per exercise performance relative to the last time. This would have a lot of confounding factors: personal differences and skill level let some add more weight with less effort. There would also be the problem of missing data (what is the effort on your first workout, or when you use new exercises?). This would also require users to input the number of repetitions and lifted weight for all their sets for all their exercises. Finally, this wouldn't reward users for putting in effort even on a bad day or during a plateau. I consulted sport science literature, and found a popular metric for estimating effort is called "RPE" (Rate of Perceived Exertion). RPE is a scale from 1-10, where 1 is almost no effort, 5 is moderate effort, and 10 is maximal effort. RPE has the downside of not being objective, as we are relying on a person's subjective perception of their effort. Research by Monica et al. found that "The odds of underestimating RPE for an exerciser were 3.67 times greater than a non-exerciser" \cite{RPE_estimations}. This could be offset by reducing the RPE that beginner's estimate, but this was out of scope for this project. I am not particularly worried about intentional dishonesty with regards to RPE, as users can lie about the number of sets they complete, how much they can lift, how often they train, etc. I decided to use RIR (repetitions in reserve), which is an inverse of RPE as it seemed easier to understand than RPE. This article \cite{rir} made a good case for measuring total training stress through average RIR and total sets. I couldn't find any specific equations that combined the two, so I created this equation, which increases as average RIR decreases and total sets increase.

\begin{algorithm}
  $granted\_attacks = floor(min(0, (10 - average\_rir) * total\_sets))$
\end{algorithm}

This reduces the number of dealt attacks as you have a higher RIR, and as you do less sets. This allows users to do more sets if they don't like pushing themselves too hard, and vice versa. 

With both attack\_damage, we can the damage dealt as 

\begin{algorithm}
  $workout\_damage = attack\_damage * granted\_attacks$ 
\end{algorithm}


\subsection{Communicating app functionality to the user}

When you open the app, you initially see two buttons on the top. One is labeled as "Strengthen Character" while the other one is labeled as "Deal Damage". "Strengthen Character" lets you update your exercise personal records, which increase your character's strength. "Deal Damage" on the other hand lets you track a workout, which deals damage towards your team's enemy. I decided to label these based on the effect that these actions have in game, as opposed to "Update Exercise" and "Track Workout" which I initially used. This is because I believe that the required real life data input could be easily inferred from these screens, so it would be a good opportunity to help the understand the real life connection to the game. As an example, if a user clicks on "Strengthen Character", and they see a screen that allows them to input their exercise performances, I would hope that they understood that your in-game strength is calculated through your real life strength.


\begin{figure}[H]
    \begin{subfigure}{0.45\textwidth}
        \includegraphics[width=\textwidth]{exercise_modal.png}
        \caption{"Strengthen Character" screen} 
    \end{subfigure}
    \begin{subfigure}{0.45\textwidth}
      \includegraphics[width=\textwidth]{workout_modal.png}    
      \caption{"Deal Damage" Screen}
      \label{Track workout}
    \end{subfigure}
\end{figure}

The right-hand "Deal Damage" screen is far more verbose than the left-hand "Strengthen Character". When I was running my pilot study, I found that participants instantly understood that they were being asked to input how many repetitions of a weight they could do on a given exercies do. Users were more confused with the idea of total sets, as they weren't sure what constituted a set. Users also struggled to understand what "reps in reserve" are. I added the explanations above each value to help explain these.

I tried to give the user instantenous satisfying and informative feedback. When a user closes the "Strengthen Character" screen after adding exercises, their attack damage changes. Depending on the new attack damage, the character also changes. I wanted everyone to experience this when for tracking tracking their first exercise(s), As long as you track a single exercise and make your attack non-zero, you will get a new character. 

\begin{figure}[H]
    \begin{subfigure}{0.45\textwidth}
      \includegraphics[width=\textwidth]{noob_user_page.png}    
      \caption{What the user sees initially (before tracking any exercises)}
      \label{Track workout}
    \end{subfigure}
    \begin{subfigure}{0.45\textwidth}
        \includegraphics[width=\textwidth]{apprentice.png}
        \caption{What the user sees after tracking 1 or more exercises} 
    \end{subfigure}
\end{figure}

I also tried to make tracking workouts feel satisfying. After saving the workout, you initiate a fight with your team's enemy. You get to see the animations of your character attack, and the enemy take damage, one hit at a time. This is skippable, as it might become boring for long-time users. Killing the enemy will also trigger a dying animation from the enemy.

\begin{figure}[H]
    \centering
    \includegraphics[width=1.0\linewidth]{attack.png}    
    \caption{
      An active battle. There were some visual bugs that I didn't have time to fix.
    }
\end{figure}

I also created a tutorial that explains this idea further. This tutorial is shown when you initially launch the app. In order to demonstrate the connection with real world strength and in-game strength, I simultaneously change real world strength inputs, and the visible in-game characters.

\begin{figure}[H]
    \begin{subfigure}{0.45\textwidth}
      \includegraphics[width=\textwidth]{noob.png}    
      \caption{The first visible character and exercise personal bests}
      \label{Track workout}
    \end{subfigure}
    \begin{subfigure}{0.45\textwidth}
        \includegraphics[width=\textwidth]{elite.png}
        \caption{The "elite" character, unlocked through high real world strength} 
    \end{subfigure}
\end{figure}

In order to demonstrate how workouts deal damage towards an enemy that you fight with your friends, I show two different characters take turns fighting a monster, where the second player finishes the monster off.

\begin{figure}[H]
    \begin{subfigure}{0.45\textwidth}
      \includegraphics[width=\textwidth]{upper_attack.png}    
      \caption{The first player deals initial damage to the enemy}
      \label{Track workout}
    \end{subfigure}
    \begin{subfigure}{0.45\textwidth}
        \includegraphics[width=\textwidth]{lower_attack.png}
        \caption{The second player then kills the wounded monster} 
    \end{subfigure}
\end{figure}



































































































\chapter{Implementation}
\subsection{Server/Database}
The server is built with a technology called Postgraphile GraphQL. GraphQL is a query language similar to REST, except GraphQL requests require you to pass JSON keys which you require the values for. This is what a simple GraphQL request might look like: 

\begin{lstlisting}[language=python, caption={An example GraphQL query fetching information about user "orek" and their messages }, label=lst:callahan]
  query{
    user(username: "orek"){
      groupname
      createdAt
      updatedAt
      chatMessagesByUsername{
        nodes{
          textContent
          createdAt
        }
      }
    }
  }
\end{lstlisting}

\begin{lstlisting}[language=python, caption={Server response to the above query}]
  {
    "data": {
      "user": {
        "groupname": "Team Public",
        "createdAt": "2021-02-18T20:30:55.784356+02:00",
        "updatedAt": "2021-02-18T20:30:55.784356+02:00",
        "chatMessagesByUsername": {
          "nodes": [
            {
              "textContent": "Good day today, right?",
              "createdAt": "2021-02-18T20:30:55.784356+02:00"
            }
          ]
        }
      }
    }
  }
\end{lstlisting}
This query format has many benefits: a client would never overfetch or underfetch. In a standard REST setup, the /user endpoint might return user metadata and /messages might return a users messages. This would mean that the client would have to send two requests, which would be slower for both the client and the server. One solution might be setting up a single endpoint, /user\_messages, which returns both user metadata and user messages. This would have the downside of overfetching: whenever we needed just the user metadata, the server would still have to do extra work to fetch messages, leading to a slower query. GraphQL only requires one request to get exactly the data that you need, no more, and no less. The data also comes in the exact shape that you request it in, which makes mapping the data on the client easy. 

This sounds good in practice, but GraphQL can often perform poorly. This was brought to my attention by this video where Awad talks about the Waterfall problem. This is characterized by GraphQL resolvers sending needlessly many database queries to fetch data. In order to implement nested resolvers (such as fetching the child books from the parent library) you need to define a function which returns children based on parent values (such as fetching all books whose foreign key matches the parent's ID). In this case, we make one query to fetch library information, and another one to fetch all books. But what if we fetch all pages of each book? This would lead to potentially thousands of database queries, as each book would call it's child resolver function fetching it's pages. This could be executed in a single SQL query by using joins.

Luckily there is a library for this. Instead of issuing an SQL query for every single child in the query, Postgraphile acts as a middleware, converting GraphQL requests into a single SQL query. This has huge performance benefits over the afformentioned manual implementation, as only one SQL query is required, and SQL's optimization can be harnessed better (such as through indexes). Another huge benefit of Postgraphile is that it analyzes your databases, and generates your entire query API from analyzing attributes and relations in your database. A lot of my time usually goes in to writing, debugging and maintaining basic CRUD server logic. When the database becomes complicated, changing the structure and relationships of tables can require a lot of work to change on the server. Postgraphile removes all this effort. Postgraphile does require some configuration, as we usually want to grant and limit data access depending on user, and internal logic beyond simply reading and writing data. 


\subsection{Security}
I am used to implementing security logic in the middleware, and not the database, so this was hard for to understand and implement. 

Removing read/write queries from the GraphQL API for specific fields and tables is very simple:

\begin{lstlisting}[language=SQL, caption={This comment tells Postgraphile to omit password fields} ]
comment on column "user".password is E'@omit';
\end{lstlisting}

The difficult part was implementing per user security. When a user signs up, their username is stored along with their salted and hashed password. They receive a JWT token containing their username and a expiry date, which is symmetrically encrypted by a secret key (to prevent anyone from issuing access tokens). 
\begin{figure}[H]
    \centering
    \includegraphics[width=1.0\linewidth]{authentication.png}    
    \caption{
  Once the user has stored the JWT token, this is how authentication works
    }
\end{figure}

Now that the database can authenticate users, it still needs to check whether or not a particular user has rights to perform an operation. This is implemented through row level security policies. Row level security policies only allow operations if the affected row(s) evaluate to true for the supplied boolean expression.


\begin{lstlisting}[language=SQL, caption={Row level security policies for accessing chat messages}, ]
CREATE POLICY chat_message_update ON "chat_message" FOR update to query_sender USING (username = (select username from active_user()));
CREATE POLICY chat_message_delete ON "chat_message" FOR delete to query_sender USING (username = (select username from active_user()));
CREATE POLICY chat_message_create ON "chat_message" FOR insert to query_sender with check (username = (select username from active_user()) and groupName = (select groupName from active_user()));
CREATE POLICY chat_message_select ON "chat_message" FOR select to query_sender using (groupName = (select groupName from active_user()));
\end{lstlisting}

The following code only allows a user to delete or update their own messages. Insert's are only allowed if the user also belongs to the group the message is sent to. Finally, any user in the group is allowed to read their group's messages. This combined with the earlier authentication protects unpermitted data access on the database itself. Although verbose, this creates consistent rules that can be written once and enforced everywhere, and prevents bugs that cause data leaks (for instance, sending all user data instead of filtering just the authenticated user's data, as this wouldn't be permitted by the RLS policies).

Finally, I will discuss password management. When a user creates an account by sending an encrypted POST request, I hash and salt the password and store the username in plaintext.

\begin{lstlisting}[language=SQL, caption={Hashing and salting done with pgcrypto}]
insert into "user"(username, password) values (username, crypt(password, gen_salt('bf')));
\end{lstlisting}

Then, when a user requests a JWT token, I simply hash their input password again, and compare it to the stored one:
\begin{lstlisting}[language=SQL, caption={Hashing and salting done with pgcrypto}]
if authenticated_user.password = crypt(input_password, authenticated_user.password) then
  #grant JWT
  #...
\end{lstlisting}

\subsection{Complex database logic}
With the current setup, we can execute simple authenticated rule-based CRUD operations. Usually we want more complex logic. One solution would be to replace the automatically generated queries and mutations with custom written ones that also include side effects. This would go against Postgraphile's philosophy of reducing the amount of source code that needs to be written. I found the best solution to introduce side effects was to use database triggers. Database triggers are functions that execute when rows are specified rows are modified. Triggers can alter the row that caused the trigger to row, as well as execute SQL that reads/writes from other tables/rows. 


\begin{lstlisting}[language=SQL, caption={Definition of a trigger function which sets the rows' "updated\_at" column to be the current time, and a trigger which calls the function on a user whenever the user is updated.}]
CREATE OR REPLACE FUNCTION trigger_set_timestamp()
RETURNS TRIGGER AS $\dollar$$\dollar$ 
BEGIN
  NEW.updated_at = NOW();
  RETURN NEW;
END;
$\dollar$$\dollar$ LANGUAGE plpgsql;
CREATE TRIGGER set_timestamp
BEFORE UPDATE ON "user"
FOR EACH ROW
EXECUTE PROCEDURE trigger_set_timestamp();
\end{lstlisting}

Triggers reduce the amount of code that needs to be written, as I only need to specify what should be done after a write operation, as Postgraphile handles the initial CRUD operation which causes side effects to trigger. Traditional side effects that are written also might need to be written more than once: if a user is updated from two different endpoints, both would need to manually update the user's timestamp. Triggers need to be written once, and will always run regardless of what query or SQL statement triggered them.

\subsection{Subscriptions}
In order to implement real time events on the server side, I used Postgraphile's built in subscriptions. Client subscribers listen on a "topic" which is a string representing a real time event source. A topic "blog\_posts" might send events notify listeners with a link to the new post whenever a new blog was published. In my case, I had topics for every team, which were formatted as "Event\_{Team Name}" (such as "Event\_Dream Team"). The Postgres database alerts Postgraphile when a topic has updated with the pg\_notify() function, along with data about the event itself. Postgraphile then notifies event listeners, along with the information about the event to the event to the listeners. This event can be a message, a workout, or a new exercise personal record. The client is responsible for presenting the different events. 

\subsection{Other server features}
My server is also responding for sending a participant information/ethics consent sheet. This is a simple React App, containing information about the study, data usage, etc. which requires the user to enter the username which they would like to use and to read and consent to all points. When the form is submitted, the username is saved in ethics consent table in the database. The presence of your username in the ethics consent table is required to proceed with the sign up, in which case the app opens your browser to let you sign in. Submitting the form opens the app back up, where you should be signed up.

\subsection{Deploying}
I originally intended to deploy both my database and server on Heroku, as I was familiar with both. I realized that Heroku's database wouldn't fit my use case: you are not allowed to create roles. My security policy is built on having a super user (which only I have access to) and a restricted database user which represents someone executing a request, which I call "query\_sender". It's important that Postgraphile forwards the SQL queries as a restricted user, as restricted users are subject to row level security policies which protect unauthorized access. I considered hosting the database and server on AWS, but Postgraphile does not support subscriptions there (due to the serverless infrastructure). I went with a Heroku server and an AWS RDS database (which supports roles, even on the free tier.) It took some work for me to get SSL properly working as I am new to using AWS. Locally, simple CRUD queries only take 10-20ms to handle, and more complex ones such as fetching users by group, and all events by users take 50ms, while on Heroku, this is 60-100ms for simple queries, and 100-200 for more complex ones. I believe that this is partially because of the latency from Heroku to AWS, but I am still happy with the performance, especially for a free server and database.

\subsection{Strength calculations}
A key part of my app is the connection between in game strength and real life strength. Strength is hard to calculate, as it depends on factors such as the exercise, the weight used, number of repetitions, the bodyweight of the user, etc. Although formulas exist for popular exercises (such as squat, bench, and deadlift in powerlifting) I wanted the app to include more exercises, especially bodyweight ones during COVID. I did some research, and couldn't find any API's that handled these calculations, but I did find websites which provided them. I contacted the owners of the sites, and they weren't willing to provide API access. I asked the owner of strengthlevel.com if they would mind me scraping their site, and they didn't. The web client contacts an API to make these requests, and by tracking my network requests when I submit the form, I can copy and analyze how my request parameters are sent. I then made the exercise request generalizable (for instance, by changing the exercise which I submitted, "barbell-squat") to the variable exerciseName. The node.js server didn't allow for the Strength Level request to be made, as it wasn't encrypted. After doing some research, I found two solutions: one would be to disable this restriction and to send the request without any encryption. This is a bad idea, as the request contains some sensitive information (exercise performance, identified gender, weight). This would also remove warnings when sending unencrypted requests to others hosts. The latter solution involved getting the public key (.pem file) of Strength Level, and encrypting the request with it. This was harder, as I didn't have experience with this, but I'm glad I learned to do it properly (as I also needed to configure Amazon's public key for database requests).

Most of the request parameters were self explanatory (exercise weight, bodyweight, repetitions, etc.) as it was obvious that they were just numbers. The exercise name parameter required more work to generalize, as this was a string which had to be in the small subset of supported exercises. I scraped the names of this by copying the HTML of the exercise browser, and stripping everything but the names. This was initially enough information to set up relative strength calculations. Months later I tracked a bodyweight exercise, and found that the request didn't work. This is because bodyweight exercise have a different query format. Instead of requiring lifted weight as a parameter, they require "additional\_weight" and "is\_bodyweight". To accomodate for this, I added an extra field to all exercises (is\_bodyweight), scraped all exercises tagged as such, and set is\_bodyweight to true for all those exercises. I also had to refactor the request sent to Strength Level depending on if the exercise was a bodyweight one or not. After fixing this bug, it was very easy to add a "bodyweight only" filter to the exercise search in my app, which hopefully made my exercise. 

I accidentally mislabeled two exercises as bodyweight, which lead to one user getting a crash. My server threw an error, as Strength Level didn't give a valid response from the malformed request, which lead the user to receive an error aswell. The client would try to resend it whenever they relaunced the app, which lead to more crashes. I fixed these exercises in the database, and told the user to remove their cache (which stopped the user from trying to send a malformed request).

I wanted groups to be able to search exercise updates by user for particular battles against monsters, or all battles. One simple solution would be to select based on timestamp (if a battle against a monster started 2 days ago, select all exercise updates where the "created\_at" attribute is less than 2 days from now). I decided not to do this, as this would be very slow since the timestamp is not a ordering attribute. Selecting all matching exercise updates would require searching all user exercise updates. Instead I decided to add nullable foreign key attributes, battle\_number and group\_name which denote the battle in which this exercise was tracked, and the group in which the user belonged to at the time. This metadata is not sent by the client, but read and written by a database trigger after the user inserts an exercise update. This trigger might not do anything, as the battle\_number and group\_name returned from the query might be null.

\begin{lstlisting}[language=SQL, caption={This selects the group and battle of the user that created this exercise update}, label=lst:trigger_set_metadata]
select group_name into NEW.group_name from "user" where username = NEW.username;
select battle_number into NEW.battle_number from "group" where name = NEW.group_name;

before insert on "user_exercise"
\end{lstlisting}

Using foreign keys allows us to construct queries GraphQL which fetch exercise updates based on group or specific battle, as Postgraphile recognizes relations. Loading the group's name into the exercise update also makes it easy to set up subscription updates. After the group name has been loaded, I read the groupname and alert subscribers of that group of the exercise update event.

Workout calculations

I implemented the same logic for updating workouts, but for workouts you must be a part of group, and you must have an active battle (this is true if you have two or more members). Tracking workouts makes you deal damage towards your enemy, which doesn't make sense if you don't have an enemy. On the other hand, you can strengthen your character without needing an active enemy. I calculate the number of granted hits using a simple generated column:

\begin{lstlisting}[language=SQL, caption={hits is generated from the user's input (average_rir and sets)}, label=lst:generated_column]
hits integer GENERATED ALWAYS as ((10 - average_rir) / 10.0 * sets) stored
\end{lstlisting}

I also needed to calculate the damage that this workout dealt, and to subtract the dealt damage from the current enemy, for which I needed to implement a trigger that activates whenever a workout is created. This trigger first calculates the total damage that this workout dealt by multiplying the number of hits with the attack damage of this user. I then subtract the current health from the group's enemy. Finally, I check if the health of the enemy is 0 or less, in which case I progress the group forward by one level, giving them a new stronger enemy at full health.
























































































\section{Guidance}
You can't talk about everything. Cover the high level first, then cover important, relevant or impressive details.


\section{General points}
These points apply to the whole dissertation, not just this chapter.


\subsection{Figures}
\emph{Always} refer to figures included, like Figure \ref{fig:relu}, in the body of the text. Include full, explanatory captions and make sure the figures look good on the page.
You may include multiple figures in one float, as in Figure \ref{fig:synthetic}, using \texttt{subcaption}, which is enabled in the template.


\subsection{Equations}

Equations should be typeset correctly and precisely. Make sure you get parenthesis sizing correct, and punctuate equations correctly 
(the comma is important and goes \textit{inside} the equation block). Explain any symbols used clearly if not defined earlier. 

For example, we might define:
\begin{equation}
    \hat{f}(\xi) = \frac{1}{2}\left[ \int_{-\infty}^{\infty} f(x) e^{2\pi i x \xi} \right],
\end{equation}    
where $\hat{f}(\xi)$ is the Fourier transform of the time domain signal $f(x)$.

\subsection{Algorithms}
Algorithms can be set using \texttt{algorithm2e}, as in Algorithm \ref{alg:metropolis}.

\chapter{Evaluation} 
How good is your solution? How well did you solve the general problem, and what evidence do you have to support that?

\section{Guidance}
\begin{itemize}
    \item
        Ask specific questions that address the general problem.
    \item
        Answer them with precise evidence (graphs, numbers, statistical
        analysis, qualitative analysis).
    \item
        Be fair and be scientific.
    \item
        The key thing is to show that you know how to evaluate your work, not
        that your work is the most amazing product ever.
\end{itemize}

\section{Evidence}
Make sure you present your evidence well. Use appropriate visualisations, reporting techniques and statistical analysis, as appropriate.

If you visualise, follow the basic rules, as illustrated in Figure \ref{fig:boxplot}:
\begin{itemize}
\item Label everything correctly (axis, title, units).
\item Caption thoroughly.
\item Reference in text.
\item \textbf{Include appropriate display of uncertainty (e.g. error bars, Box plot)}
\item Minimize clutter.
\end{itemize}

See the file \texttt{guide\_to\_visualising.pdf} for further information and guidance.



%==================================================================================================================================
\chapter{Conclusion}    
Summarise the whole project for a lazy reader who didn't read the rest (e.g. a prize-awarding committee).
\section{Guidance}
\begin{itemize}
    \item
        Summarise briefly and fairly.
    \item
        You should be addressing the general problem you introduced in the
        Introduction.        
    \item
        Include summary of concrete results (``the new compiler ran 2x
        faster'')
    \item
        Indicate what future work could be done, but remember: \textbf{you
        won't get credit for things you haven't done}.
\end{itemize}

%==================================================================================================================================
%
% 
%==================================================================================================================================
%  APPENDICES  

\begin{appendices}



\begin{itemize}
\item
  Copies of ethics approvals (required if obtained)
\item
  Copies of questionnaires etc. used to gather data from subjects.
\item
  Extensive tables or figures that are too bulky to fit in the main body of
  the report, particularly ones that are repetitive and summarised in the body.

\item Outline of the source code (e.g. directory structure), or other architecture documentation like class diagrams.

\item User manuals, and any guides to starting/running the software.

\end{itemize}

\textbf{Don't include your source code in the appendices}. It will be
submitted separately.

\end{appendices}

%==================================================================================================================================
%   BIBLIOGRAPHY   

% The bibliography style is abbrvnat
% The bibliography always appears last, after the appendices.

\bibliographystyle{abbrvnat}

\bibliography{l4proj}


\end{document}

